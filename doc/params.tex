\input _macros

\title{Parameters of \ass\ Cores}

\chapter{Sizes}

\ass\ is agnostic to three size parameters:

\medskip

\li \wsi, the ``word size'', or, equivalently, the size of an address on the
platform. This parameter is equivalent to a number of other important
parameters:

\smallskip

\lili The width of the registers; and
\lili The minimum width of the memory/cache data bus.

\smallskip

\li \asi, the ``addressable unit'', which represents the granularity of memory
acccesses. This parameter cannot meaningfully be larger than \wsi. Most modern
platforms use ``octets'' (8 bits).

\li \isi, the ``instruction size'' (the letter ``J'' chosen because it is the
amount of bits by which one must \underbar{J}ump to the next
instruction)---presently, only a 16-bit \isi is defined, with a 32-bit being
considered experimental. (Some other important parameters, such as the number
of registers, may vary between instruction sizes; the details are not yet
complete.)

\medskip

From these parameters, the following are also defined:

\medskip

\li {\tt \ist\ = \isi\ / \asi}, rounded up, is the ``instruction step size'',
which is the number of addressable units by which \reg{PC} must be advanced to
point to the next instruction. For example, if \isi\ is 16 and \asi\ is 8, then
\ist\ is 2.

\li {\tt \wst\ = \wsi\ / \asi}, rounded up, is the ``word step size'', which is
the number of addessable units by which a pointer (address) must be incremented
to point to the next word (relevant, for example, for the Data Transfer
\code{XF} instruction). For example, if \wsi\ is 64 and \asi\ is 8, then \wst\
is 8.

\chapter{Naming}

The name of a single \ass\ core instantiation is as follows:

{\noindent\qquad\tt\ass\ \isi/\wsi}

(TODO: \asi\ is left unstated.)

\maybye
