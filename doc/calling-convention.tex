\input _macros

\title{Calling Convention}

\chapter{Register Assignment}

\vbox{
	\offinterlineskip
	\def\tspace#1{height#1&\omit&&\omit&&\omit&\cr}
	\def\inrule{\tspace{1pt}\noalign{\hrule}\tspace{1pt}}
	\def\tcell#1{\vtop{\hsize=5in\noindent#1}}
	\halign{&\vrule#&\strut\quad#\hfil\quad\cr
		\noalign{\hrule}
		\tspace{2pt}
		&Register&&Purpose&&Saved By&\cr
		\tspace{2pt}
		\noalign{\hrule}
		&\reg{R0}&&\tcell{
			Program Counter (\reg{PC}) or Instruction Pointer (\reg{IP}). This
			is fixed micro\-arch\-it\-ect\-ur\-al\-ly---instruction fetch
			occurs from the address specified in this register. It can be read
			and written like any other, but note that there is an
			instruction-size increment applied (approximately) after fetch,
			such that \reg{PC} is already pointing at the next instruction (or
			word after this instruction), which tends to combine conveniently
			with automodification. As it occurs during the writeback phase,
			stores targeting \reg{PC} will ignore this increment and set the
			register directly. \insn{JAL}, similarly, reads the post-fetch
			incremented value, which is what is stored in the link register.
		}&&Caller&\cr
		\inrule
		&\reg{R1}&&\tcell{
			Conventionally, the Stack Pointer (\reg{SP}). This points into
			``the stack'', a region of memory which ``grows down'', in that
			elements on top of the stack are at numerically lower addresses. A
			push is accomplished by \code{XF *-\reg{SP}, SRC}, and a pop by
			\code{XF dst, *\reg{SP}+}, for some targets \code{src} and
			\code{dst}.
		}&&Callee&\cr
		\inrule
		&\reg{R2}&&\tcell{
			Conventionally, the Frame Pointer (\reg{FP}). This points to a
			compiler-known location in the (live part of) the stack which
			contains the activation record. As such, its value should not be
			changed during the function's execution. Local variables are
			accessed by offset from this register.
		}&&Callee&\cr
		\inrule
		&\reg{R3}&&\tcell{
			Conventionally, the Return Continuation address (\reg{RA}). This
			should generally be the ``link'' register of \insn{JAL}. After
			restoring all callee-saved registers (including this one),
			returning from the function consists of a jump back to this
			address, as with \code{MOV PC, RC}.
		}&&Callee&\cr
		\inrule
		&\reg{R4} -- \reg{R9}&&\tcell{
			Conventionally, these six registers are the Argument Registers
			(\reg{A0} -- \reg{A5}). They are all caller-saved, meaning the
			callee is free to do as it wishes; conventionally, the arguments
			are passed in them, and return values may be read from them.
			Compilers are free to use as many registers per argument as they
			see fit; extra arguments are usually spilled onto the top of the
			stack, where they remain at function entry.
		}&&Caller&\cr
		\inrule
		&\reg{R10} -- \reg{R15}&&\tcell{
			Conventionally, these six registers are the Temporary Registers
			(\reg{T0} -- \reg{T5}). They are callee-saved, which means their
			value must be restored at the end of the routine. Otherwise, they
			have no prescribed purpose.
		}&&Callee&\cr
		\tspace{2pt}
		\noalign{\hrule}
	}
}
\bigskip
\chapter{Function {\chapterit Pro Forma}}

\section{Call}

To call a function, assume we have its absolute address in a pseudo-register
``\reg{FA}''. \reg{FA} can be any correctly used register; for example:

\li It can be any one of the \reg{A0} -- \reg{A5} Argument Registers, as long
as that register isn't live at this time; or

\li It can be any Temporary Register \reg{T0} -- \reg{T5}, or even the Return
Continuation \reg{RA}, so long as these have been appropriately saved by the
caller.

First, all live Argument Registers \reg{A0} -- \reg{A5} that are {\it not}
\reg{FA} are spilled; as they are caller-saved, the callee is free to use any
and all of them as it wishes, so their state cannot be guaranteed if this is
not done. These registers can be live for multiple reasons, but the most
typical is that they are arguments passed from the present function's caller
which will be used later, or they are temporaries in local use.

\startblock
	\# Assume \reg{A0} -- \reg{A2} are live
	XF *-\reg{SP}, \reg{A2}
	XF *-\reg{SP}, \reg{A1}
	XF *-\reg{SP}, \reg{A0}
\endblock

(The actual order of spilling does not matter, so long as it is consistent.)

After this, loading the address register is simple enough: \code{LI \reg{FA},
func} for a function starting at the label \code{func} would suffice in all
cases, though this embeds the full word of \code{func}'s value into the
instruction stream (\insn{LI} compiles down to \code{XF \reg{FA}, *\reg{PC}+}).
For functions near the beginning of the address space, \insn{SI} will also work
(as in \code{SI \reg{FA}, func}), but only if \code{func} is small enough to be
a ``small'' immediate. Thus, absent considerable code constraints, \insn{LI} is
safer.

\startblock
	LI \reg{FA}, func
\endblock

The call sequence, following this load, is one instruction. Note that \reg{RA}
will be clobbered, which means it {\it must} be spilled, but this is usually
handled in the prologue (see below).

\startblock
	JAL \reg{RA}, \reg{FA}
\endblock

Afterward, agreed-upon values are live in the Argument Registers, which can be
used in other computations. After they are no longer live, the previous values
can be restored:

\startblock
	\# Assume \reg{A0} is an output, and \reg{T0} has already been spilled to
	\# store it
	MOV \reg{T0}, \reg{A0}
	XF \reg{A0}, *\reg{SP}+
	XF \reg{A1}, *\reg{SP}+
	XF \reg{A2}, *\reg{SP}+
\endblock

\section{Prologue}

On entry to the function, one of the first things to reliably do is spill
\reg{RA}, the Return Continuation, where the function will jump. Doing so first
tends to emulate the ``CALL'' instruction of other architectures. While not
strictly necessary, as a liveness analysis could reveal, doing so consistently
is helpful more often than not.

\startblock
	XF *-\reg{SP}, \reg{RA}
\endblock

After this, the Frame Pointer \reg{FP} can be updated. This is, again, not
going to be needed in functions without local arguments, but those are assumed
to be in the minority.

\startblock
	XF *-\reg{SP}, \reg{FP}
	MOV \reg{FP}, \reg{SP}
\endblock

With the Frame Pointer now updated to the Stack Pointer, the compiler can
statically index from \reg{FP} to access local variables in a known
configuration.

\section{Epilogue}

After all other callee-saved registers are restored, {\it except} for \reg{FP},
\reg{SP}, and \reg{RA}, restoration proceeds more or less in the reverse of the
prologue.

\startblock
	XF \reg{FP}, *\reg{SP}+
	XF \reg{RA}, *\reg{SP}+
\endblock

With all callee-saved registers now restored, all that remains is to jump back
to the caller's next instruction. Fortunately, this is in \reg{RA}:

\startblock
	MOV \reg{PC}, \reg{RA}
\endblock

\maybye
