\input _macros

\title{Instructions}

\def\ins{{\tt I}}
\def\rg#1,#2.{$[#1:#2]$}
\def\ix#1{$[#1]$}

\chapter{ASS16}

The following is the encoding of ASS16, a 16-bit instruction-word ISA. Values
are given in their natural (big-endian) representation, such that in
\code{0xABCD}, \code{A} is the most significant digit.

\bigskip
{
	\halign{#\qquad&&\hbox to2em{\hss#\hss}\cr
		Hex&A&&&&B&&&&C&&&&D&&&\cr
		Bit&a&a&a&a&b&b&b&b&c&c&c&c&d&d&d&d\cr
		Index&15&14&13&12&11&10&9&8&7&6&5&4&3&2&1&0\cr
	}
}
\bigskip

C syntax is used in explications; familiarity with it is assumed. Ranges
\rg a,b. are sorted with $a > b$, and such that it is inclusive on $b$ but
exclusive on $a$ (thus indices are in the interval $[b, a)$, equivalently for
index $i: b \le i < a$). This notation is {\it not} consistent with Verilog
(which uses both indices inclusive), but {\it is} consistent with Python
(albeit reversed), and conveniently means that the size of the extracted
bitstring is $a - b$.

The following sections assume that the instruction word is \ins.

ASS16 has 16 General-Purpose Registers (GPRs). Where an instruction encodes 4
bits for a ``register'', and implicitly refers to its value within the
description, this should be taken to be the value stored {\it within} the
register so numbered. All registers can be accessed in this fashion, but
\reg{R0} is the Program Counter (\reg{PC}), so its value will be used to fetch
the next instruction---thus, programs with linear control flow should not use
this register. (It can be used, of course, to branch or jump.) See the calling
convention document for details.

The GPRs are not to be confused with the SRs, System Registers, which are far
less frequently accessed; see that section for details.

ASS is entirely independent of word size; it is assumed that the registers and
memory bus can be of any size, though 16, 32, and 64 bit are envisioned.
Instructions below may need to refer to ``the word size in bytes'', but this is
done per-instantiation, and care has generally been taken to ensure that
instruction codings do not vary with word size. For a given instantiation, the
device is specified as ``ASS16/W'', where W is the word size---for example,
``ASS16/32'' would be a 32-bit bus, 16-bit instruction ASS CPU.

\section{Bitwise Instructions}

\noindent\ins\rg 16,12. \code{== 0b0001}

\li \ins\rg 4,0. (\reg{D}) Destination Register, LHS (0--15)
\li \ins\rg 8,4. (\reg{S}) Source Register, RHS (0--15)
\li \ins\rg 12,8. (\code{M}) Truth Table Number (0--15)

Cf. \underbar{https://en.wikipedia.org/wiki/Truth\_table\#Binary\_operations} .

The value \code{M} can be treated as a Look-Up Table (LUT), using a bit from
\reg{D} as the most-significant and a corresponding bit from \reg{S} as the
least-significant of an index into the 4-bit table. For convenience, the
functions are reproduced below:

\bigskip
{
	\tabskip=3pt
	\halign{&{\bf #}\hss&\hbox to4em{= #\hss}\cr
		False&0&NOR&1&Not-Conv Impl&2&Not D&3\cr
		Not Impl&4&Not S&5&XOR&6&NAND&7\cr
		AND&8&XNOR&9&S&10&Impl&11\cr
		D&12&Conv Impl&13&OR&14&True&15\cr
	}
}
\bigskip

Of these, the most useful are:

\li {\bf AND, OR, XOR, NAND, NOR, XNOR}: Traditional bitwise building blocks.

\li {\bf Not D, Not S}: Complement, either in-place or of \reg{S}.

\li {\bf S}: Implements \insn{MOV}, by copying \reg{S} to \reg{D}.

\li {\bf D}: No-op.

\li {\bf True, False}: Load all-ones or all-zeros.

The rest are variations on material implication, and useful in limited
settings.

\section{Arithmetic Instructions}

\noindent\ins\rg 16,12. \code{== 0b0010}

\li \ins\rg 4,0. (\reg{D}) Destination Register, LHS (0--15)
\li \ins\rg 8,4. (\reg{S}) Source Register, RHS (0--15)
\li \ins\rg 11,8. (\code{M}) Operation Number (0--7)
\li \ins\ix{11} (\code{Si}) Source Immediate

If \code{Si} is set, the literal 4-bit value of the \reg{S} field is to be used
as \reg{S} in the operation. Otherwise, as usual, the 4-bit field is a register
number, and that register's value is to be used as \reg{S}.

\bigskip
{
	\offinterlineskip
	\halign{&\vrule#&\strut\quad#\hfil\quad\cr
		\noalign{\hrule}
		height2pt&\omit&&\omit&&\omit&\cr
		&{\bf Mode}&&{\bf Operation}&&{\bf Note}&\cr
		height2pt&\omit&&\omit&&\omit&\cr
		\noalign{\hrule}
		height2pt&\omit&&\omit&&\omit&\cr
		&0&&Add ($D = D + S$)&&&\cr
		&1&&Sub ($D = D - S$)&&&\cr
		&2&&Shl ($D = D << S$, fill $0$)&&&\cr
		&3&&Shr ($D = D >> S$, fill $0$)&&&\cr
		&4&&Asr ($D = D >> S$, fill MSB of $D$)&&&\cr
		&5&&Mul ($D = D * S$)&&*&\cr
		&6&&Div ($D = D / S$, integer part)&&*&\cr
		&7&&Mod ($D = D \% S$)&&*&\cr
		height2pt&\omit&&\omit&&\omit&\cr
		\noalign{\hrule}
	}
}
\bigskip

Entries marked with ``*'' are available only on models with a hardware
multiplier (see \reg{SR 7}).

\section{Comparison Instructions}

\noindent\ins\rg 16,12. \code{== 0b0011}

\li \ins\rg 4,0. (\reg{D}) Destination Register, LHS (0--15)
\li \ins\rg 8,4. (\reg{S}) Source Register, RHS (0--15)
\li \ins\ix{8} (\code{E}) Equality Test
\li \ins\ix{9} (\code{G}) Greater Test
\li \ins\ix{10} (\code{S}) Sign Flag
\li \ins\ix{11} (\code{I}) Invert Flag

The internal test is true if either:

\li \code{E} is set and $D = S$; or
\li \code{G} is set and $D > S$, where ``$>$'' is sign-respecting if \code{S}
is set.

The final test is the internal test, or its negation if \code{I} is set.

If the final test is false, $0$ is written to \reg{D}. A nonzero value is
written if the test is true, canonically $1$.

The resulting table of operations, with some redundancy, is as follows, with
``Above''/``Below'' being interpreted as unsigned comparison, and
``Greater''/``Less'' as signed:

\bigskip
{
	\tabskip=3pt
	\halign{&{\bf #}\hss&\hbox to4em{= #\hss}\cr
		False&0&Equal&1&Above&2&Above/Equal&3\cr
		False&4&Equal&5&Greater&6&Greater/Equal&7\cr
		True&8&Not Equal&9&Below/Equal&10&Below&11\cr
		True&12&Not Equal&13&Less&14&Less/Equal&15\cr
	}
}
\bigskip

\section{Conditional Instructions}

\noindent\ins\rg 16,12. \code{== 0b1000}

\li \ins\rg 8,0. (\code{S}) Signed Offset, Instructions ($-128$--$127$)
\li \ins\rg 12,8. (\reg{C}) Comparison Register (0--15)

If the value of \reg{C} is nonzero, \reg{PC} is offset by \code{S} from the
{\it next} instruction. The value of \code{S} is interpreted to be in {\it
instruction words} (typically 2 bytes for ASS16).

\section{Jump and Link}

\noindent\ins\rg 16,11. \code{== 0b10010}

\li \ins\rg 4,0. (\reg{L}) Link Register (written, next \reg{PC}) (0--15)
\li \ins\rg 8,4. (\reg{P}) Program Register (read, new \reg{PC})) (0--15)
\li \ins\rg 11,8. Reserved

Using the \reg{PC} of the next instruction, assign \reg{L} to \reg{PC}, then
assign \reg{PC} to \reg{P}. This implements a rudimentary ``CALL'' instruction;
for suggestions on implementing a calling convention, see the separate
document (which, for example, prescribes \reg{R3} as the typical Link
Register).

\section{Subword Operation}

\noindent\ins\rg 16,11. \code{== 0b10011}

\li \ins\rg 4,0. (\reg{D}) Destination Register (0--15)
\li \ins\rg 8,4. (\code{Bi}) Byte Index (0--15)
\li \ins\rg 11,8. (\code{B}) Bytes (0--7)

Selects bytes out of a word-sized register. The lowest-order byte accessed in
\reg{D} is \code{Bi}, and \code{B} bytes are accessed from it, shifted right by
\code{Bi}. If $B = 0$, 8 bytes are selected instead.

This is approximately equivalent to the following assembly, with \code{B} in
\reg{A0} and \code{Bi} in \reg{A1}:

\startblock
	XF *-\reg{SP}, \reg{T0}
	XF *-\reg{SP}, \reg{T1}
	\# If $B = 0$, set \code{B} to 8.
	BIT \reg{T0}, \reg{T0}, 0  \# Load zeros
	MOV \reg{T1}, \reg{A0}
	CMP.EI \reg{T1}, \reg{T0}
	JC \reg{T1}, .window
	SI \reg{A0}, 8
{\leftskip=2em.window:\par}
	MOV \reg{T0}, \reg{A0}
	SHL \reg{T0}, \$3  \# 8 bits to a byte
	SI \reg{T1}, 1
	SHL \reg{T1}, \reg{T0}
	SUB \reg{T1}, \$1
	SHL \reg{A1}, \$3
	SHL \reg{T1}, \reg{A1}
	BIT \reg{D}, \reg{T1}, 8  \# AND
	SHR \reg{D}, \reg{A1}
{\leftskip=2em.cleanup:\par}
	XF \reg{T1}, *\reg{SP}+
	XF \reg{T0}, *\reg{SP}+
\endblock

\section{System Registers}

\noindent\ins\rg 16,13. \code{== 0b101}

\li \ins\rg 8,0. (\reg{SR}) System Register (0--255)
\li \ins\rg 12,8. (\reg{R}) General Register (0--15)
\li \ins\ix{12} (\code{W}) Write Flag

Read from \reg{SR} to \reg{R}, or write from \reg{R} to \reg{SR}.

The System Registers have a separate namespace from the General-Purpose
Registers; some are architectural constants, while others have meaning specific
to the hardware. The defined ones are as follows:

\bigskip
{
	\offinterlineskip
	\long\def\tcell#1{\vtop{\baselineskip=12pt\lineskip=1pt\lineskiplimit=0pt\hsize=3.75in\noindent#1}}
	\def\tspace#1{height#1&\omit&&\omit&&\omit&&\omit&\cr}
	\def\trule#1{
		\tspace{#1}
		\noalign{\hrule}
		\tspace{#1}
	}
	\def\tpad{\hskip2pt\relax}
	\halign{&\vrule#&\strut\tpad#\hfil\tpad\cr
		\noalign{\hrule}
		\tspace{2pt}
		&{\bf SR}&&{\bf R/W}&&{\bf Name}&&{\bf Description}&\cr
		\trule{2pt}
		&0&&R&&Model/Stepping&&\tcell{%
			\li 0: Unknown or prototype
			\li Otherwise: reserved
		}&\cr
		\trule{2pt}
		&1&&R&&Word Size&&In bits.&\cr
		\trule{2pt}
		&2&&RW&&Exception Handling Routine&&\tcell{%
			\li Bit 0: Exceptions enabled. Automatically cleared when entering
			the routine.
			\li All others: loaded into \reg{PC} when an exception or interrupt
			occurs; bit 0 of PC is set to 0. (See also \reg{SR 8}.)
		}&\cr
		\trule{2pt}
		&3&&W&&Power State&&\tcell{%
			\li Bit 0: Halt
			\li All other bits: reserved, write 0
		}&\cr
		\trule{2pt}
		&4&&RW&&Instructions Retired&&\tcell{%
			Counts continuously, incremented when an instruction is retired.
			Writing simply sets the counter. Initialized to 0 on reset.
		}&\cr
		\trule{2pt}
		&5&&RW&&Cycles Elapsed&&\tcell{%
			Counts the number of clock cycles elapsed, starting at 0 on reset.
			Writing simply sets the counter.
		}&\cr
		\trule{2pt}
		&6&&R&&Core Frequency&&\tcell{%
			Real frequency in $s^{-1}$ of the clock used for \reg{SR 5},
			divided by 1024 to produce a value in kibiHertz.
		}&\cr
		\trule{2pt}
		&7&&R&&ISA Extensions&&\tcell{%
			\li Bit 0: Hardware multiplier available
			\li All other bits: reserved
		}&\cr
		\trule{2pt}
		&8&&RW&&Exception Continuation&&\tcell{%
			Contains the ``next \reg{PC}'' value before the Exception Handling
			Routine (\reg{SR 2}) was entered. In \insn{JAL} terms, \reg{SR 2}
			is the \reg{P} ``Program Register'', and this register is the
			\reg{L} ``Link Register''.
		}&\cr
		\trule{2pt}
		&9&&RW&&Exception Cause&&\tcell{%
			\li Bit 0: Registered Exception. If set, the remaining bits
			(logically right shifted by one) have an ISA-determined
			interpretation:
			\par\vtop{ \advance\leftskip by\parindent\noindent
				\li 0: Divide by zero
				\li 1: Illegal instruction
				\li 2: Supervisor request
			}
			\hang Otherwise (if Bit 0 is clear), the entire register contains
			the physical MMIO address of the device responsible for raising a
			hardware interrupt.
		}&\cr
		\trule{2pt}
		&10&&W&&Unofficial Debug Port&&\tcell{%
			Writing a byte will write it to stdout in most current simulators.
			Hardware implementors are allowed to ignore writes to this
			register, but it may be of some use to expose them via an external
			debugging interface.
		}&\cr
		\trule{2pt}
		&192 -- 255&&-&&Vendor Extensions&&Reserved for vendors.&\cr
		\tspace{2pt}
		\noalign{\hrule}
	}
}
\bigskip

\section{Small Immediate}

\noindent\ins\rg 16,12. \code{== 0b1100}

\li \ins\rg 4,0. (\reg{D}) Destination Register (0--15)
\li \ins\rg 12,4. (\code{I}) Immediate (0--255)

Stores the unsigned value \code{I} into the register \reg{D}.

\section{Data Transfer}

\noindent\ins\rg 16,14. \code{== 0b01}

\li \ins\rg 4,0. (\reg{D}) Destination Register (0--15)
\li \ins\rg 8,4. (\reg{S}) Source Register (0--15)
\li \ins\rg 10,8. (\code{Dm}) Destination Mode (0--3)
\li \ins\ix{10} (\code{Di}) Destination Indirect
\li \ins\rg 13,11. (\code{Sm}) Source Mode (0--3)
\li \ins\ix{13} (\code{Si}) Source Indirect

If the corresponding indirect bit is set, the register contents are used as a
memory address, and the appropriate load/store behavior is done.

The ``mode'' determines how the register is changed before and after the
operation:

\li 0b00: No change.
\li 0b01: Autoincrement; increment the register by the number of bytes in a
word after the operation.
\li 0b10: Autodecrement: decrement the register by the number of bytes in a
word before the operation.
\li 0b11: Autopostdecrement: decrement the register by the number of bytes in a
word after the operation.

The operations, in order, are a load from \reg{S} (which may be indirect), and
a store to \reg{D} (which also may be indirect). It is possible to use any of
the modes with non-indirect access, which (for example) may be used to
implement counting loops.

\bye
